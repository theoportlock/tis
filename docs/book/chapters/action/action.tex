\chapter{Action}
\section{How do we act}
Decision field theory is a method of mathematically modelling the change in utility when comparing multiple choices \cite{busemeyer2002survey}.

\section{How do machines act}
% talk if statements
% say that the action comes from humans interperating the computational results of machines

\section{The random-pianist method}
% compare to neurological likeness
% talk about the fact that people only do what they have observed before
% think about free will here (or maybe in the intro?)


\section{The Upgrade workflow}
No delayed combinations
only reinforce if you see it muliple times
I = input that does not result from an action
A = input that results from an action
R = random input (or choice of previous set: pseudorandom)
[] = combination expansion
Both the action and the actions effects should be recognised by combinations

I1, R1        -->    A1[I1, R1]
I2, A1, R2    -->    A2[I2, A1, R2]
I3, A2, R3    -->    A3[I3, A2, R3]
I4, A3, R4    -->    A4[I4, A3, R4]

Aims:
1: Learn actions from inputs
2: Decrease entropy from the UEI
3: Learn inputs from inputs

Whats being combined for each timepoint:
Random doping
I1, R1
I2, R2, A1[I1, R1]
I3, R3, A2[I2, R2, A1[I1, R1]]
...

actions included
I1, A1
I2, A1(I1)
I3, A2(I2), A2(A1(I1)), A2(I2 + A1(T1))

W/O random doping
I1
I2, A1[I1]
I3, A2[I2, A1[I1]]
...

expanded W/O random doping
I1
I2, A1(I1)
I3, A2(I2), A2(A1(I1)), A2(I2 + A1(T1))
...

translated:
new input
the action resulting from the second input
the action resulting from the action resulting from the first input
the action resulting from a combination of the second input and the action resulting from the first input 

R1, I1, A1
