\chapter{Introduction}
\section{Morality}
Morality is the set of principles that define right and wrong actions.
A moral agent is an individual that acts according to a moral system.
In a moral system, an event is considered morally virtuous (good) or reprehensible (bad) if its consequences increase, or decrease the probability of a moral end respectively.

\subsection{Moral relationships - Interaction between two moral systems}
Moral ends vary between individuals.
The moral ends of individuals can be determined from a record of past actions.
When individual moral agents interact, one of three possible interactions can be observed.
Firstly, a rejection interaction manifests as an assumption that one another's actions are bad (minus - minus). 
Secondly, an unbalanced interaction is a consequence of the willingness of only one moral agent to encourage the others pursuit of their respective moral systemi (plus - minus).
Lastly, an acceptance interaction is the understanding that the ends described by a moral system converges to the point that their continued actions weighted by their estimated probability is similar to those actions taken by the moral system of another individual (plus - plus).

\subsection{Moral consensus - Majority moral opinion}
As it is with the legal system, an estimation of common "good" is essential for collective progress.
The search for this consensus is done at a personal or a collective level when we vote, poll, or protest. 
An estimation of common "good" is also necessary for collective arbitration of an action; a field referred to as descriptive or comparative ethics.
Once a consensus is established, the application of the golden rule, the principle of treating others as you wish to be treated, forms the basic principles of justice and social contract.
Communication of intention can often be reciprocal, that is to say one individual has and understanding of the others understanding of your understanding (and so on).
An understanding of the consequences of breaking this rule encourages the enforcement of a social contract.
% The lesser the amount of knowledge surrounding subject matter, the greater the room for reasonable disagreement. 
% Factual consensus has no claim to moral authority

\subsection{Moral evolution - How morality changes over time}
% A couple of case studies (slavery, womens/gay rights etc.)
% Conjecture and refutation
% Judgement of people from different timepoints
% time dependant moral graphs
% axiom pca plot
Consensus morality changes over time.
Moral opinions of an action can be changed by a greater understanding of probabilities of consequences that arise after the action.
Moral re-evaluation is also possible through the revelation of inconsistencies.
The cause of this can be due to incomplete data relating to a moral question or cognitive dissonance.

\section{The state of current majority moral opinion}
Although the secularisation of western countries and adoption of enlightenment ideals is often considered a loss of moral congruency, the opposite is most likely as a consequence of interreligious moral alignment.
Based on polling data, core moral principles include:
% Define progress in terms of morality

\subsection{How is society improving?}
With respect to the current moral consensus, there has been significant moral progress \cite{pinker2011better}.

\subsection{How is society deteriorating?}
% Talk about media and public perception
% List areas that are in decline and tie with how 

\section{Fallibilities and limitations of human decision making}
In order to reverse these forms of societal deterioration, it is important to look at the origins.
\subsection{Bias and Discrimination}
\subsection{Misinformation}
\subsection{Missing information}
\subsection{The scale of human computation}
With the advancements in electronic data storage, the availability of data brings with it the potential for more accurate predictive tools than any point previously.
However, the current restraints of human computational load limits the rate at which these tools can be utilised.

\section{Alternative decision making algorithms}
Currently, there has been no success in the construction of an artificial moral agent that mitigates the real issues in human moral decision making process.
% the case for alternate tools for moral decision making
\subsection{Deep learning}
\subsection{HTM}
Numenta \cite{numentahome}, alpha go, etc
\subsection{How to align intelligent systems with consensus moral beliefs}
As with humans, if an agent has greater intelligence then there is a greater importance that the agent has a moral system that aligns with the core values that supports the current majority moral opinion.
Transparency of decision making process is cruicial for the public arbitration of artificial decision making algorithms>

\section{AI safety and responsibility}
\subsection{The "off button" problem}
\subsection{The paperclip problem}
% Also called the stamp maker problem
% the assurance of alignment of the upgrade with consensus core moral beliefs.
% an appeal to the upgrade for aligning with the preferences of peoples moral systems

\section{Project aims}
Current attempts have fallen short at achieving these ends and resulted in hyperspecific reward seeking behaviour.
The aim of this study is to leverage the advancements of computing to create a system that will make more moral choices than those made by a human.
Separate from traditional neural network architectures, Hirachical Temporal Memory (HTM) is a technique developed by Numenta that more closely resembles the connections between axons in the brain.
In this study, the guiding principles of HTM combined with a novel system for self validation will be designed to evaluate \emph{ab initio} learning of a new machine learning model.
Python will be used principally for development, testing, and validation.
% factors that affect this probability to succeed
% passion
% time
% expertise
% manpower
% justify the writing of this document 
% Probablilty machines
