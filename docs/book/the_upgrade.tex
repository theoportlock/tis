\documentclass{book}
\usepackage[hidelinks]{hyperref}
\usepackage[left=4cm, right=3cm, top=3cm, bottom=2cm]{geometry}
\usepackage{apacite}
\bibliographystyle{apacite}
\usepackage{booktabs}
\usepackage{fancyhdr}
\fancyhead{}
\fancyhead[LE, RO]{\slshape \leftmark}
\fancyfoot[C]{\thepage}
\pagestyle{fancy}
%\includeonly{
%    chapters/introduction/introduction,
%    }

\title{A Combinatorics Approach to Pattern Recognition of Bitarray Datastreams}
\author{Theo Portlock}

\begin{document}
\frontmatter
\maketitle

\chapter{Abstract}
In our lives, we try to make decisions that increase the possiblity of a moral outcome.
We judge others by their moral objectives and their adherance to them.
Identifying, understanding, and mitigating the flaws of human thought process and decision making is essential for human progress.
Despite much progress made in the field of artificial intelligence, a tool that makes better, more informed moral decisions than a human has not been developed and is the aim of this study.
A novel approach that involves the field of combinatorics to bitarray datastreams with a likeness to the neuronal communication in the brain was developed to achieve these ends.
The possibility of succeeding is incalcuable but increases with effort, time, and expertise.
However, if good progress towards these aims are made, the consequences will pave the way for the future of a more moral society.

\cleardoublepage
\phantomsection
\addcontentsline{toc}{chapter}{Table of Contents}
\tableofcontents

\cleardoublepage
\phantomsection
\addcontentsline{toc}{chapter}{List of Figures}
\listoffigures

\cleardoublepage
\phantomsection
\addcontentsline{toc}{chapter}{List of Tables}
\listoftables

\mainmatter
\chapter{Introduction}
\section{Morality}
Morality is the set of principles used by individuals to decide right and wrong actions.
A moral agent is an individual that acts within a moral system.
In a moral system, an event is considered morally virtuous (good) or reprehensible (bad) if its consequences increase, or decrease the probability of a moral end respectively.
A moral agent that acts predictably based on a set of moral rules then it is principled.

\subsection{Majority moral opinion}
Moral ends vary between individuals.
As it is with the legal system, an estimation of common "good" is essential for collective progress.
The search for this consensus is done at a personal or a collective level when we vote, poll, or protest. 
When individual moral agents interact, one of three possible combinations can be observed.
Firstly, a rejection of moral systems manifests as a bipartisan assumption of one another's actions are bad. 
Secondly, an unbalanced interaction is a consequence of the willingness of only one moral agent to encourage the others pursuit of their respective moral system.
Lastly, a mutual acceptance of moral systems is the understanding that the ends described by a moral system converges to the point that their continued actions weighted by their estimated probability is similar to those actions taken by the moral system of another individual.
An estimation of common "good" is also necessary for collective arbitration of an action; a field referred to as descriptive or comparative ethics.
Once a consensus is established, the application of the golden rule, the principle of treating others as you wish to be treated, forms the concepts of justice and social contract.
Communication of intention can often be reciprocal, that is to say one individual has and understanding of the others understanding of your understanding (and so on).
An understanding of the consequences of breaking this rule encourages the enforcement of a social contract.
% The lesser the amount of knowledge surrounding subject matter, the greater the room for reasonable disagreement. 
% Factual consensus has no claim to moral authority

\subsection{How morality changes}
% A couple of case studies (slavery, womens/gay rights etc.)
% Conjecture and refutation
% Judgement of people from different timepoints
% time dependant moral graphs
% axiom pca plot
Consensus morality changes over time.
Moral opinions of an action can be changed by a greater understanding of probabilities of consequences that arise after the action.
Moral re-evaluation is also possible through the revelation of inconsistencies.
The cause of this can be due to incomplete data relating to a moral question or cognitive dissonance.

\section{The state of current majority moral opinion}
Although the secularisation of western countries and adoption of enlightenment ideals is often considered a loss of moral congruency, the opposite is most likely as a consequence of interreligious moral alignment.
Based on polling data, core moral principles include:
% Define progress in terms of morality

\subsection{How is society improving?}
With respect to the current moral consensus, there has been significant moral progress  
% Steven pinkers 57 graphs

\subsection{How is society deteriorating?}
% Talk about media and public perception
% List areas that are in decline and tie with how 

\section{Fallibilities and limitations of human decision making}
\subsection{Bias and Discrimination}
\subsection{Misinformation}
\subsection{Missing information}
\subsection{The scale of human computation}
With the advancements in electronic data storage, the availability of data brings with it the potential for more accurate predictive tools than any point previously.
However, the current restraints of human computational load limits the rate at which these tools can be utilised.

\section{Alternative decision making algorithms}
Currently, there has been no success in the construction of an artificial moral agent that mitigates the real issues in human moral decision making process.
\subsection{Deep learning}
\subsection{HTM}
Numenta \cite{numentahome}, alpha go, etc
\subsection{How to align intelligent systems with consensus moral beliefs}
As with humans, if an agent has greater intelligence then there is a greater importance that the agent has a moral system that aligns with the core values that supports the current majority moral opinion.
Transparency of decision making process is cruicial for the public arbitration of artificial decision making algorithms>

\section{AI safety and responsibility}
\subsection{The "off button" problem}
\subsection{The paperclip problem}
% Also called the stamp maker problem
% the assurance of alignment of the upgrade with consensus core moral beliefs.
% an appeal to the upgrade for aligning with the preferences of peoples moral systems

\section{Project aims}
Current attempts have fallen short at achieving these ends and resulted in hyperspecific reward seeking behaviour.
The aim of this study is to leverage the advancements of computing to create a system that will make more moral choices than those made by a human.
Separate from traditional neural network architectures, Hirachical Temporal Memory (HTM) is a technique developed by Numenta that more closely resembles the connections between axons in the brain.
In this study, the guiding principles of HTM combined with a novel system for self validation will be designed to evaluate \emph{ab initio} learning of a new machine learning model.
Python will be used principally for development, testing, and validation.
% factors that affect this probability to succeed
% passion
% time
% expertise
% manpower
% justify the writing of this document 

\chapter{Datastreams}
\section{Sources of data}
% compare to how humans intereperate biological data in brains
% compare time dependent/independent learning

\section{Binary conversion}
% talk about the virtues of working with binary data

\section{Sparcity}
% talk about self controling for levels of sparcity

\section{Tandem data input}
% ensure that all data streams are consistent - shouldn't be essential

\chapter{Combinations}
\section{Chapter introduction}
\subsection{Classical combinatorics}
\subsection{The problem of scale}
\section{The combinations array}
\section{Slicing the combinations array}
SCALE ADJUSTMENT
TIME ADJUSTMENT
SKEW ADJUSTMENT
\subsection{Combinations of combinations}
PAIR FINDING
SET THEORY
PROBABILITIES OF PROBABILITIES
\section{Chapter summary}

\chapter{Memory}
\section{Storage}
\section{Prediction}
%^M = a,b,ab,d,ad,bd,abd
%I = a,b
%Ih = c,d
%^C(I) = a,b,ab
%^C(Ih) = c,d,cd
%^C(I)+C(Ih) = a,b,ab,c,d,cd
%^M | (C(I)+C(Ih)) = ad,bd,abd
%^^C(C(I)+C(Ih)) = [a],[b],[a,b],[ab],[a,ab],[b,ab],[a,b,ab],[c],[c,a],[c,b],[c,a,b],[c,ab],[c,a,ab],[c,b,ab],[c,a,b,ab].... 
%extract the paired combinations that weren't found in either of the lists
%^^C(C(I)+C(Ih))-(C(C(I))+C(C(Ih)) = [a,c],[a,d],[a,c,d],[a,cd],[a,c,cd],[a,d,cd],[b,c]...
%^^C(C(I)+C(Ih))-(C(C(I))+C(C(Ih))-notpairs = [a,c],[a,d],[a,cd],[b,c],[b,d],[b,cd],[ab,c],[ab,d],[ab,cd]
%^agg.sum(C(C(I)+C(Ih))-(C(C(I))+C(C(Ih))-notpairs) = ac,ad,acd,bc,bd,bcd,abc,abd,abcd
%^agg.sum(C(C(I)+C(Ih))-(C(C(I))+C(C(Ih))-notpairs) | M = ad,bd,abd
%^^agg.sum(C(C(I)+C(Ih))-(C(C(I))+C(C(Ih))-notpairs) | M = [a,d],[b,d],[a,b,d]
%^I = a,b

%Move to factorial based function generation for the slicing of memory arrays - for speed
%This would mean that loading the full memory array into RAM would be unnecessary, just use function to look for bits in memory binary file

%weighted probability - giving a higher score prediction for a better match
%the democratic effect

	
\section{Persistance of activation}
\section{Edge contraction}
\section{Delay function}
\section{Transfer and Storage}
\section{Memory merging}

\chapter{Action}
\section{How are decisions made}
Decision field theory is a method of mathematically modelling the change in utility when comparing multiple choices \cite{busemeyer2002survey}.
% how do you determine if an action happens beacuse of you or someone else?
Currently, machines do not act this way.
% talk if statements
% say that the action comes from humans interperating the computational results of machines

\section{The random-pianist method}
% Difference between action and prediction
% compare to neurological likeness
% talk about the fact that people only do what they have observed before
% think about free will here (or maybe in the intro?)

\section{The Upgrade workflow}
% consequences of random actions on input are put into memory
% input and actions are memorised
% an input causes a precition - AB -> ABCT
%% if prediction was different, then action would be different
% The prediction is used as input to 
% when you predict an action you activate it
% How to measure the extent of influence of the new input from the consesequences of the actions of the old input - repeat
% The contribution of random to the action determines the level of understanding?
% If random get's it right then
No delayed combinations
only reinforce if you see it muliple times
I = input that does not result from an action
A = input that results from an action
R = random input (or choice of previous set: pseudorandom)
[] = combination expansion
Both the action and the actions effects should be recognised by combinations

I1, R1        -->    A1[I1, R1]
I2, A1, R2    -->    A2[I2, A1, R2]
I3, A2, R3    -->    A3[I3, A2, R3]
I4, A3, R4    -->    A4[I4, A3, R4]

Aims:
1: Learn actions from inputs
2: Decrease entropy from the UEI
3: Learn inputs from inputs

Whats being combined for each timepoint:
Random doping
I1, R1
I2, R2, A1[I1, R1]
I3, R3, A2[I2, R2, A1[I1, R1]]
...

actions included
I1, A1
I2, A1(I1)
I3, A2(I2), A2(A1(I1)), A2(I2 + A1(T1))

W/O random doping
I1
I2, A1[I1]
I3, A2[I2, A1[I1]]
...

expanded W/O random doping
I1
I2, A1(I1)
I3, A2(I2), A2(A1(I1)), A2(I2 + A1(T1))
...

translated:
new input
the action resulting from the second input
the action resulting from the action resulting from the first input
the action resulting from a combination of the second input and the action resulting from the first input 

R1, I1, A1

\chapter{Runtime}
\section{Chapter introduction}
% PRINCIPLE OF MAXIMAL UTILISATION
% In order to run at a constant operations per unit time, input must be substituted by prediction when it contains fewer active bits and vica versa.

\chapter{Testing}
\section{Chapter introduction}
This chapter will focus on some of the more well known problems that face current machine learning models. 
\section{Test 1 - Mathematics}
\subsection{Addition}
\section{Test 2 - Natural language modelling}
\subsection{Text prediction}
\section{Test 3 - Signal processing}
\subsection{Voice recognition}
\section{Test 4 - Image classification}
\subsection{Medical diagnosis}


\backmatter
\bibliography{library}
\chapter{Appendix}
\section{Chapter introduction}

\end{document}
