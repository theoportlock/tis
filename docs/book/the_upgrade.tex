\documentclass{book}
\usepackage[hidelinks]{hyperref}
\usepackage[left=4cm, right=3cm, top=3cm, bottom=2cm]{geometry}
\usepackage{apacite}
\bibliographystyle{apacite}
\usepackage{booktabs}
\usepackage{fancyhdr}
\fancyhead{}
\fancyhead[LE, RO]{\slshape \leftmark}
\fancyfoot[C]{\thepage}
\pagestyle{fancy}
%\includeonly{
%    chapters/introduction/introduction,
%    }

\title{A Combinatorics Approach to Pattern Recognition of Bitarray Datastreams}
\author{Theo Portlock}

\begin{document}
\frontmatter
\maketitle

\chapter{Abstract}
In our lives, we try to make decisions that increase the possiblity of a moral outcome.
We judge others by their moral objectives and their adherance to them.
Identifying, understanding, and mitigating the flaws of human thought process and decision making is essential for human progress.
Despite much progress made in the field of artificial intelligence, a tool that makes better, more informed moral decisions than a human has not been developed and is the aim of this study.
A novel approach that involves the field of combinatorics to bitarray datastreams with a likeness to the neuronal communication in the brain was developed to achieve these ends.
The possibility of succeeding is incalcuable but increases with effort, time, and expertise.
However, if good progress towards these aims are made, the consequences will pave the way for the future of a more moral society.

\cleardoublepage
\phantomsection
\addcontentsline{toc}{chapter}{Table of Contents}
\tableofcontents

\cleardoublepage
\phantomsection
\addcontentsline{toc}{chapter}{List of Figures}
\listoffigures

\cleardoublepage
\phantomsection
\addcontentsline{toc}{chapter}{List of Tables}
\listoftables

\mainmatter
\chapter{Introduction}
\section{Morality}
Morality is the set of priciples used by individuals that define right and wrong actions.
A moral agent is an individual that acts within a moral system.
An event is concidered moraly virtuous (good) with respect to a moral system if its consequences increase the probability of a moral end.
An event is concidered moraly reprehensible (bad) with respect to a moral system if its consequences decrease the probability of a moral end.
A moral system can contain multiple ends but often, a combination of these ends weighted by personal importance can be assumed to be an end in of itself.

\subsection{Majority moral opinion}
Moral ends vary between individuals.
When individual moral agents interact, one of three possible combinations can be observed.
Firstly, a mutual rejection of moral systems can be defined as the assumption of one anothers actions are bad. 
Secondly, an unbalanced interaction is a consequence of the willingness of only one moral agent to encourage the others persuit of their respective moral system.
Lasty, a mutual acceptance of moral systems is the understanding that the ends described by a moral system is aligned to the point that their continued actions weighted by their estimated probability is similar to those actions taken by the moral system of another individual.
As with the legal system, an estimation of common "good" is necessary for collective arbritration of an action.
Once a consensus is established, the application of the golden rule, the principle of treating others as you wish to be treated, forms the idea of justice and social contract.
The search for this consensus is done at a personal scale during adolesence, or at a collective scale when we vote, polls, or protest.
Communication of intention can often be reciprocal, that is to say one individual has and understanding of the others understanding of your understanding and so on.
An understanding of the consequences of breaking this rule encourages the enforcement of a social contract

\subsection{How morality changes}
% A couple of case studies (slavery, womens/gay rights etc.)
% Conjecture and refutation
% Judgement of people from different timepoints
% time dependant moral graphs
% axiom pca plot
Consensus morality changes over time.
Moral opinions can be changed with varying probabilities.
The only means of moral re-evaluation is through the revelation of inconsistencies.
This can be due to incomplete data relating to a moral question or cognitive dissonance.

\section{Progress}
% Define progress in terms of morality

\subsection{How is society improving?}
% Steven pinkers 57 graphs

\subsection{How is society deteriorating?}
% Talk about media and public perception
% List areas that are in decline and tie with how 

\section{The fallibility of human decision making}
\subsection{Bias}
\subsection{Misinformation}
\subsection{Missing information}

\section{Alternative decision making algorithms}
\subsection{Deep learning}
\subsection{HTM}
Numenta \cite{numentahome}, alpha go, etc
\subsection{How to align intelligent systems with consensus moral beliefs}

\section{AI saftey and responsibility}
\subsection{The "off button" problem}
\subsection{The paperclip problem}
% Also called the stamp maker problem
% the assurance of alignment of the upgrade with consensus core moral beliefs.
% an appeal to the upgrade for aligning with the preferences of peoples moral systems

\section{Project aims}
% factors that affect this probability to succeed
% passion
% time
% expertise
% manpower
% justify the writing of this document 

\chapter{Datastreams}
\section{Sources of data}
% compare to how humans intereperate biological data in brains
% compare time dependent/independent learning

\section{Binary conversion}
% talk about the virtues of working with binary data

\section{Sparcity}
% talk about controling the levels of sparcity

\section{Tandem data input}
% ensure that all data streams are consistent - shouldn't be essential

\chapter{Combinations}
\section{Chapter introduction}
\subsection{Classical combinatorics}
\subsection{The problem of scale}
\section{The combinations array}
\section{Slicing the combinations array}
SCALE ADJUSTMENT
TIME ADJUSTMENT
SKEW ADJUSTMENT
\subsection{Combinations of combinations}
PAIR FINDING
SET THEORY
PROBABILITIES OF PROBABILITIES
\section{Chapter summary}

\chapter{Memory}
\minitoc
\section{Chapter introduction}
\section{Persistance of activation}
\section{Delay function}
\section{Transfer and Storage}
\section{Summary}

\chapter{Action}
\section{How do we act}
Decision field theory is a method of mathematically modelling the change in utility when comparing multiple choices \cite{busemeyer2002survey}.

\section{How do machines act}
% talk if statements
% say that the action comes from humans interperating the computational results of machines

\section{The random-pianist method}
% compare to neurological likeness
% talk about the fact that people only do what they have observed before
% think about free will here (or maybe in the intro?)


\section{The Upgrade workflow}
No delayed combinations
only reinforce if you see it muliple times
I = input that does not result from an action
A = input that results from an action
R = random input (or choice of previous set: pseudorandom)
[] = combination expansion
Both the action and the actions effects should be recognised by combinations

I1, R1        -->    A1[I1, R1]
I2, A1, R2    -->    A2[I2, A1, R2]
I3, A2, R3    -->    A3[I3, A2, R3]
I4, A3, R4    -->    A4[I4, A3, R4]

Aims:
1: Learn actions from inputs
2: Decrease entropy from the UEI
3: Learn inputs from inputs

Whats being combined for each timepoint:
Random doping
I1, R1
I2, R2, A1[I1, R1]
I3, R3, A2[I2, R2, A1[I1, R1]]
...

#actions included
#I1, A1
#I2, A1(I1)
#I3, A2(I2), A2(A1(I1)), A2(I2 + A1(T1))

W/O random doping
I1
I2, A1[I1]
I3, A2[I2, A1[I1]]
...

expanded W/O random doping
I1
I2, A1(I1)
I3, A2(I2), A2(A1(I1)), A2(I2 + A1(T1))
...

translated:
new input
the action resulting from the second input
the action resulting from the action resulting from the first input
the action resulting from a combination of the second input and the action resulting from the first input 

R1, I1, A1

\chapter{Runtime}
\section{Chapter introduction}
% PRINCIPLE OF MAXIMAL UTILISATION
% In order to run at a constant operations per unit time, input must be substituted by prediction when it contains fewer active bits and vica versa.

\chapter{Testing}
\section{Chapter introduction}
This chapter will focus on some of the more well known problems that face current machine learning models. 
\section{Test 1 - Mathematics}
\subsection{Addition}
\section{Test 2 - Natural language modelling}
\subsection{Text prediction}
\section{Test 3 - Signal processing}
\subsection{Voice recognition}
\section{Test 4 - Image classification}
\subsection{Medical diagnosis}


\backmatter
\bibliography{library}
\chapter{Appendix}
\minitoc
\section{Chapter introduction}

\end{document}
