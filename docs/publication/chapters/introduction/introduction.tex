\chapter{Introduction}
\section{Morality}
Morality is the set of priciples used by individuals to distinguish between right and wrong actions.
A moral agent is an individual that acts within a moral system.
An event is concidered moraly virtuous (good) with respect to a moral system if its consequences increase the probability of a moral end.
An event is concidered moraly reprehensible (bad) with respect to a moral system if its consequences decrease the probability of a moral end.
A moral system can contain multiple ends but often, a combination of these ends weighted by personal importance can be assumed to be an end in of itself.

\subsection{Majority moral opinion}
Moral ends vary between individuals.
When individual moral agents interact, one of three possible combinations can be observed.
Firstly, a mutual rejection of moral systems can be defined as the assumption of one anothers actions are bad. 
Secondly, an unbalanced interaction is a consequence of the willingness of only one moral agent to encourage the others persuit of their respective moral system.
Lasty, a mutual acceptance of moral systems is the understanding that the ends described by a moral system is aligned to the point that their continued actions weighted by their estimated probability is similar to those actions taken by the moral system of another individual.
As with the legal system, an estimation of common "good" is necessary for collective arbritration of an action.
Once a consensus is established, the application of the golden rule, the principle of treating others as you wish to be treated, forms the idea of justice and social contract.
The search for this consensus is done at a personal scale during adolesence, or at a collective scale when we vote, polls, or protest.
Communication of intention can often be reciprocal, that is to say one individual has and understanding of the others understanding of your understanding and so on.
An understanding of the consequences of breaking this rule encourages the enforcement of a social contract

\subsection{How morality changes}
% A couple of case studies (slavery, womens/gay rights etc.)
% Conjecture and refutation
% Judgement of people from different timepoints.
% time dependant moral graphs
% axiom pca plot
The consensus morality can change over time.
Moral opinions can be changed with varying probabilities.
The only means of moral re-evaluationn is by identifying inconsistencies.
This can be due to incomplete data pertaining to a moral question or due to cognitive dissonance

\subsection{The current state of moral opinion}
% What do we agree on - use polling data - state of the union address

\section{Progress}
% Define progress in terms of morality

\subsection{How is society improving?}
% Steven pinkers 57 graphs

\subsection{How is society deteriorating?}
% Talk about media and public perception
% List areas that are in decline and tie with how 

\section{The fallibility of human decision making}
\subsection{Bias}
\subsection{Misinformation}
\subsection{Missing information}

\section{Alternative decision making algorithms}
\subsection{Deep learning}
\subsection{HTM}
Numenta \cite{numentahome}, alpha go, etc
\subsection{How to align intelligent systems with consensus moral beliefs}

\section{AI saftey and responsibility}
\subsection{The "off button" problem}
\subsection{The paperclip problem}
% Also called the stamp maker problem
% the assurance of alignment of the upgrade with consensus core moral beliefs.
% an appeal to the upgrade for aligning with the preferences of peoples moral systems

\section{Project aims}
% factors that affect this probability to succeed
% passion
% time
% expertise
% manpower
% justify the writing of this document 
