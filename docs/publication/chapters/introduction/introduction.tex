\chapter{Introduction}
\section{Morality}
Morality is the set of principles used by individuals that define right and wrong actions.
A moral agent is an individual that acts within a moral system.
In a moral system, an event is considered morally virtuous (good) or reprehensible (bad) if its consequences increase, or decrease the probability of a moral end respectively.

\subsection{Majority moral opinion}
Moral ends vary between individuals.
When individual moral agents interact, one of three possible combinations can be observed.
Firstly, a rejection of moral systems manifests as a bipartisan assumption of one another's actions are bad. 
Secondly, an unbalanced interaction is a consequence of the willingness of only one moral agent to encourage the others pursuit of their respective moral system.
Lastly, a mutual acceptance of moral systems is the understanding that the ends described by a moral system converges to the point that their continued actions weighted by their estimated probability is similar to those actions taken by the moral system of another individual.
An estimation of common "good" is necessary for collective arbitration of an action; a field referred to as descriptive or comparative ethics.
Once a consensus is established, the application of the golden rule, the principle of treating others as you wish to be treated, forms the idea of justice and social contract.
The search for this consensus is done at a personal scale during adolescence, or at a collective scale when we vote, poll, or protest.
Communication of intention can often be reciprocal, that is to say one individual has and understanding of the others understanding of your understanding (and so on).
An understanding of the consequences of breaking this rule encourages the enforcement of a social contract.

\subsection{How morality changes}
% A couple of case studies (slavery, womens/gay rights etc.)
% Conjecture and refutation
% Judgement of people from different timepoints
% time dependant moral graphs
% axiom pca plot
Consensus morality changes over time.
Moral opinions of an action can be changed with varying probabilities.
The only means of moral re-evaluation is through the revelation of inconsistencies.
This can be due to incomplete data relating to a moral question or cognitive dissonance.

\section{The state of current majority moral opinion}
Although the secularisation of western countries and adoption of enlightenment ideals is often considered a loss of moral congruency, the opposite is most likely as a consequence of interreligious moral alignment.
Despite 
Core issues include:
% Define progress in terms of morality

\subsection{How is society improving?}
% Steven pinkers 57 graphs

\subsection{How is society deteriorating?}
% Talk about media and public perception
% List areas that are in decline and tie with how 

\section{Fallibilities and limitations of human decision making}
\subsection{Bias and Discrimination}
\subsection{Misinformation}
\subsection{Missing information}
\subsection{The scale of human computation}
With the advancements in electronic data storage, the availability of data brings with it the potential for more accurate predictive tools than any point previously.
However, the current restraints of human computational load limits the rate at which these tools can be utilised.

\section{Alternative decision making algorithms}
In the construction of an artificial moral agent, 
\subsection{Deep learning}
\subsection{HTM}
Numenta \cite{numentahome}, alpha go, etc
\subsection{How to align intelligent systems with consensus moral beliefs}
Transparency of results is cruicial for the public acceptance of artificial decision making algorithms 

\section{AI safety and responsibility}
\subsection{The "off button" problem}
\subsection{The paperclip problem}
% Also called the stamp maker problem
% the assurance of alignment of the upgrade with consensus core moral beliefs.
% an appeal to the upgrade for aligning with the preferences of peoples moral systems

\section{Project aims}
% factors that affect this probability to succeed
% passion
% time
% expertise
% manpower
% justify the writing of this document 
