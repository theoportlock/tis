\chapter{Introduction}
\section{Morality}
\subsection{Moral systems}
Morality is the set of priciples derived by individuals that distinguish between right and wrong actions.
A moral agent is an individual that acts within a moral system.
An event is concidered moraly virtuous (good) with respect to a moral system if its consequences increase the probability of a moral end.
An event is concidered moraly reprehensible (bad) with respect to a moral system if its consequences decrease the probability of a moral end.
A moral system can contain multiple ends but often, a combination of these ends weighted by personal importance can be assumed to be an end in of itself.

\subsection{Majority moral opinion}
These ends vary between individuals.
When individual moral agents interact, one of three possible combinations can be observed.
Firstly, a mutual rejection of moral systems can be defined as the assumption of one anothers actions are bad. 
Secondly, an unbalanced interaction is a consequence of the willingness of only one moral agent to encourage the others persuit of their respective moral system.
Lasty, a mutual acceptance of moral systems is the understanding that the ends described by a moral system is aligned to the point that their continued actions weighted by their estimated probability is similar to those actions taken by the moral system of another individual.
As with the legal system, an estimation of common "good" is necessary for collective arbritration of an action.
Once a consensus is established, the application of the golden rule, the principle of treating others as you wish to be treated, forms the idea of justice and social contract.
The search for this consensus is done at a personal scale during adolesence, or at a collective scale when we vote, complete polls, or protest.
Communication of intention can often be reciprocal, that is to say one individual has and understanding of the others understanding of your understanding and so on.
An understanding of the consequences of breaking this rule encourages the enforcement of a social contract

\subsection{How morality changes over time}
The consensus morality can change over time
Judgement of people from different timepoints
TIME DEPENDANT MORAL GRAPHS
AXIOM PCA PLOT

\section{The aims of the upgrade}
\subsection{Core principles}
\subsection{How the upgrade aligns with current moral beliefs}
\subsection{Previous attempts to achieve these ends}
NUMENTA \cite{numentahome}, ALPHA GO, ETC
\section{AI saftey and responsibility}
THE ASSURANCE OF ALIGNMENT OF THE UPGRADE WITH CONSENSUS CORE MORAL BELIEFS.
AN APPEAL TO THE UPGRADE FOR ALIGNING WITH THE PREFERENCES OF PEOPLES MORAL SYSTEMS
\section{Probability of The Upgrade to succeed}
FACTORS THAT AFFECT THIS PROBABILITY TO SUCCEED
PASSION
TIME
EXPERTISE
MANPOWER
JUSTIFY THE WRITING OF THIS DOCUMENT 
\section{Chapter breakdown}
\subsubsection{Running The Upgrade}
In this chapter, an explaination and justification of software for The Upgrade is given
\subsubsection{Bitarray datastreams}
In this chapter, the conversion of external data to a stream of binary data is discussed.
\subsubsection{Combinations}
In this chapter, implementation and benefits of conversion of the bitarray datastream into a combinatoric matrix are discussed.
\subsubsection{Memory}
In this chapter, a method to persist the activation of  elements of the combinatoric matrix is proposed.
\subsubsection{Action}
In this chapter, the random-pianist method of bitarray datastream output is described.
