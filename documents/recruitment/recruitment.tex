\documentclass[12pt]{article}
\usepackage[hidelinks]{hyperref}
\usepackage[left=4cm,right=3cm,top=3cm,bottom=3cm]{geometry}
\usepackage{apacite}
\usepackage{siunitx}
\title{The Upgrade: Recruitment}
\date{\vspace{-5ex}}
%\author{Theo Portlock}
\author{\vspace{-5ex}}
\bibliographystyle{apacite}
\begin{document}
\maketitle

In our lives, we try to make decisions that increase the possiblity of a moral outcome. As humans, we are imperfect moral agents restrained by the shackles of our biological instincts. The aim of The Upgrade is to create a tool that will make more moral choices than those made by a human. The possibility of this project achieving that aim is incalculable. However, if it does, the consequences will pave the way for the future of a more moral society.

Morality is the set of priciples that distinguish between right and wrong actions. An action is concidered "good" if its consequences increase the probability of achieving a desired end. These ends vary between individuals. As it is with the legal system, an estimation of common "good" is essential for the arbritration of an action. The search for this consensus is done at a personal or a collective scale when we vote, poll, or protest. The ability to change a moral system comes only from identifying inconsitencies within it. As a consequence, shared values are refined over time. 

\section*{Project aims}
TIE IN WITH MORALS
CURRENT ATTEMPTS
NUMENTA \cite{numentahome}, ALPHA GO, ETC
AI SAFETY
THE ASSURANCE OF ALIGNMENT OF THE UPGRADE WITH CONSENSUS CORE MORAL BELIEFS.
AN APPEAL TO THE UPGRADE FOR ALIGNING WITH THE PREFERENCES OF PEOPLES MORAL SYSTEMS
FACTORS THAT AFFECT THIS PROBABILITY TO SUCCEED
PASSION
TIME
EXPERTISE
MANPOWER
JUSTIFY THE WRITING OF THIS MANUAL 

\section*{Methods}
\section*{Significance of research}
\section*{Bibliography}
\bibliography{library}
\end{document}
