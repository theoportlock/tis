\documentclass[12pt]{article}
\usepackage[hidelinks]{hyperref}
\usepackage[left=4cm,right=3cm,top=3cm,bottom=3cm]{geometry}
\usepackage{apacite}
\usepackage{siunitx}
\title{The Upgrade: Recruitment}
\date{\vspace{-5ex}}
%\author{Theo Portlock}
\author{\vspace{-5ex}}
\bibliographystyle{apacite}
\begin{document}
\maketitle


Morality is the set of priciples that distinguish between right and wrong actions. An action is concidered "good" if its consequences increase the probability of achieving a desired end. These ends vary between individuals. As it is with the legal system, an estimation of common "good" is essential for the arbritration of an action. The search for this consensus is done at a personal or a collective scale when we vote, poll, or protest.  Any Artificial General Intelligent (AGI) agent should adhere to the consensus morality of the time which it exists. Current attempts at achieving these aims have resulted in hyperspecific preprogrammed reward seeking behaviour.

\section*{Project aims}
In our lives, we try to make decisions that increase the possiblity of a moral outcome. As humans, we are imperfect moral agents restrained by the shackles of our biology. The aim of this study is to create a tool that will make more moral choices than those made by a human. 

\section*{Methods}
Hirachical temporal memory (HTM) is a technique developed by Numenta \cite{numentahome} is separate from traditional neural network architectures.  In this study, the guiding principles of HTM combined with a novel system for self validation will be used to evaluate \emph{ab initio} learning of a new machine learning model. Python will be used principally for development, testing, and validation

\section*{Significance of research}
The possibility of this project achieving its aim is incalculable. However, if it does, the consequences will pave the way for the future of a more moral society.

\section*{Bibliography}
\bibliography{library}
\end{document}
