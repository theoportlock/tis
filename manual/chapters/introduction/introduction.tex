\chapter{Introduction}
\minitoc
\section{Moral systems}
\subsection{Majority moral opinion}
Morality is the set of priciples that distinguish between right and wrong actions. An action is concidered moraly virtuous or "good" if its consequences increase the probability of achieving an end. These ends vary between individuals. As it is with the legal system, an estimation of common "good" is important for the arbritration of an action. Having a consensus of shared values permits also the use of the golden rule; the principle of treating others as you want to be treated, the basis of which forms the idea of justice. 
COMPARISON OF POLLING DATA FOR MORALS
METAETHICAL JUSTIFICATION FOR BELIEFS
METAETHICAL EVALUATION OF MORAL SYSTEMS
AN APPEAL TO THE UPGRADE FOR ALIGNING WITH THE PREFERENCES OF PEOPLES MORAL SYSTEMS
USE THE COMPARISONS TO FIND A CONSENSUS MORAL OPINION (LAWS ETC)
\subsection{How morality changes over time}
TIME DEPENDANT MORAL GRAPHS
AXIOM PCA PLOT
\section{The aims of the upgrade}
\subsection{Core principles}
\subsection{How the upgrade aligns with current moral beliefs}
\subsection{Previous attempts to achieve these ends}
NUMENTA \cite{numentahome}, ALPHA GO, ETC
\section{AI saftey and responsibility}
THE ASSURANCE OF ALIGNMENT OF THE UPGRADE WITH CONSENSUS CORE MORAL BELIEFS.
\section{Probability of The Upgrade to succeed}
FACTORS THAT AFFECT THIS PROBABILITY TO SUCCEED
PASSION
TIME
EXPERTISE
MANPOWER
JUSTIFY THE WRITING OF THIS MANUAL 
\section{Chapter abstracts}
\subsubsection{Running The Upgrade}
In this chapter, an explaination and justification of software for The Upgrade is given
\subsubsection{Bitarray datastreams}
In this chapter, the conversion of external data to a stream of binary data is discussed.
\subsubsection{Combinations}
In this chapter, implementation and benefits of conversion of the bitarray datastream into a combinatoric matrix are discussed.
\subsubsection{Memory}
In this chapter, a method to persist the activation of  elements of the combinatoric matrix is proposed.
\subsubsection{Action}
In this chapter, the random-pianist method of bitarray datastream output is described.
